% typeset: Pdftex
% Afterwards compile with pdflatex > bibtex > pdflatex > pdflatex
% TeXShop Settings... > Engine > BibTex Engine > biber
% beamer likes biber
% latex likes bibtex

% https://tex.stackexchange.com/questions/270633/beamer-and-the-pause-command
% https://tex.stackexchange.com/questions/1423/is-there-a-nice-way-to-compile-a-beamer-presentation-without-the-pauses
% Beamer presentation template for serber; created by /Users/dantopa//GitHub/genesis/scripts/new-pres.sh at 2025-01-01 07:44:34
% \documentclass[ handout ]{beamer} % for printing
\documentclass[ ]{beamer}

% Fetch home directory: make this file independent of file system
\usepackage{catchfile}
\CatchFileDef{\HomePath}{|kpsewhich -var-value=HOME}{}
% Define base paths
% relies on symlink  at /Users/dantopa/, e.g.
%   GitHub -> /Users/dantopa//repos-xiuhcoatl/github
\makeatletter
\edef\HomePath{\expandafter\zap@space\HomePath \@empty}
\makeatother

% self-locate GitHub repo
\newcommand{\pGithub} {\HomePath/GitHub/}
\newcommand{\pRepo} {\pGithub/latex-2025/}

% base files
\newcommand{\pBase} {\pTrunk/\pRepo/base}
\newcommand{\pBaseConf} {\pBase/config}
\newcommand{\pBaseCommon} {\pBaseConf/common}

% local navigation
\newcommand{\type} {pres}  		%% doc, pres
\newcommand{\flavor} {research}  	%% his, research
\newcommand{\title} {serber}  		%% project title

% enunciator enunciations
\newcommand{\enunMain}  {  **  **  **  **  **  }
\newcommand{\enunBase}  {  ==  ==  ==  ==  ==  }
\newcommand{\enunLocal}  {  --  --  --  --  --  }
\newcommand{\enunPres}  {  ++  ++  ++  ++  ++  }
\newcommand{\enunDocs}  {  @@  @@  @@  @@  @@  }

\typeout{\enunMain  Begin configuration sequence: BASE, TYPE, LOCAL}
\typeout{\enunMain  pGlobConfType = \pGlobConfType}

% Configuration hierarchy: COMMON, TYPE ( doc, pres ), LOCAL
\input{\pBaseCommon/"config-common.tex"}
\input{\pTypeCommon/"config-type.tex"}
\input{\pLocalCommon/"config-local.tex"}

% ===========================================================
% Global and Local Resource Setup
% The following lines load various global and local resource
% configurations, paths, and package lists required for the 
% document. These files are part of the shared library located
% ===========================================================

% config-common.tex

%   listings-codes.tex
%   num-components.tex
%   num-list.tex
%   packages-common.tex
%   paths-global.tex
%   paths-local.tex}
%   paths-bitbucket
%   theorems.tex

% Choose hyperlink configuration:
\input{\pGlobConfCom/"href-hidden.tex"}   % For hidden links (clean, professional)
% \input{\pGlobConfCom/"href-visible.tex"} % For visible links (debugging, drafts)

%\usepackage[printwatermark]{xwatermark}
% \newwatermark[ allpages, color=red!5, angle=45, scale=3, xpos=0, ypos=0 ]{DRAFT}

% Debugging with visible slide boundaries
% \setbeamertemplate{background canvas}[grid][ step = 1cm ]

%\usepackage[printwatermark]{xwatermark}
% \newwatermark[allpages,color=red!5,angle=45,scale=3,xpos=0,ypos=0]{DRAFT}

%   --   --   --   --   --   --   --   --   --   -- Bibliography
\input{\pGlobConfCom/"bib-config-a.tex"}
\addbibresource{\pBibs/serber.bib}
%\addbibresource{\pBibs/additional.bib}

%   --   --   --   --   --   --   --   --   --   -- Title, Author
\title[Nuclear Scission]{Nuclear Scission Energy Release}
\author[Daniel Topa]{\TopaHII \\ \TopaHIIEmail}
\institute{\missiontech} 
\date{\today}

%   --   --   --   --   --   --   --   --   --   -- Structure
\begin{document}

\begin{frame}
    \titlepage
\end{frame}

\begin{frame}
\frametitle{Compute Energy Release for Fission of \href{https://en.wikipedia.org/wiki/Uranium-235}{$^{{235}}$U}}
\begin{center}
		\href{https://www.atomicarchive.com/science/fission/index.html}{
		\begin{overpic}[ scale = 0.75 ]
			{\pLocalGraphics fission/fission-02}
			%\put(-7,-10){Compute the temperature of the bar as a function of position.}
		\end{overpic}}
\end{center}
\end{frame}

\begin{frame}[ allowframebreaks ]\frametitle{Outline}
  \tableofcontents[ hideallsubsections ]
\end{frame}


% Main content
	\input{\pSections "sec Nuclear Models"}
	%\input{\pSections "sec serber"}
	%\input{\pSections "sec bohr-wheeler"}
	%\input{\pSections "sec backup"}

% Bibliography
\begin{frame}[ allowframebreaks ]
    \frametitle{Bibliography}
    \printbibliography
\end{frame}

\begin{frame}
    \titlepage
\end{frame}

\end{document}

%\tiny
%\scriptsize
%\footnotesize
%\small
%\normalsize
%\large
%\Large
%\LARGE
%\huge
%\Huge

%\, thin space (normally 1/6 of a quad);
%\> medium space (normally 2/9 of a quad);
%\; thick space (normally 5/18 of a quad);

\begin{frame}\frametitle{Frame Title}
\begin{enumerate}
  \item 
  \item 
  \item 
\end{enumerate}
\end{frame}

\begin{frame}\frametitle{Frame Title}
\begin{equation}
  \begin{array}{ccc} 
      %
      %
      %
  \end{array}
%\label{eq:}
\end{equation}
\end{frame}

\begin{frame}\frametitle{ }
\center
  \href{url}{
  \begin{overpic}[ scale = 1.0 ]
  {\pLocalGraphics graphic-file}
    %\put(-7,-10){Auxiliary text.}
  \end{overpic}}
\end{frame}

\begin{frame}\frametitle{Frame Title}
\begin{table}[htp]
%\caption{default}
\begin{center}
  \begin{tabular}{cc}
    %
    %
    %
  \end{tabular}
\end{center}
%\label{tab:label}
\end{table}
\end{frame}
%         ---         ---         ---         ---         ---   Equations
\begin{equation}
  
\label{eq:}
\end{equation}
%
\begin{equation}
\begin{split}
  & \
\end{split} 
\label{eq:}
\end{equation}
%
\begin{equation*}
  
\end{equation*}
%
\begin{equation}
\begin{array}{ccc}
    %
  & & \
    %
  & & \
    %
\end{array}
\end{equation}
%         ---         ---         ---         ---         ---   Tables
\begin{table}[htp]
\caption{default}
\begin{center}
\begin{tabular}{ccc}
    %
  & & \
    %
  & & \
    %
\end{tabular}
\end{center}
\label{tab:}
\end{table}
%         ---         ---         ---         ---         ---   Lists
\begin{enumerate}
  \item 
  \item
  \item
\end{enumerate}
%
\begin{enumerate}
  \item 
  \begin{enumerate}
    \item 
  \end{enumerate}
  \item
  \item
\end{enumerate}
\endinput  %  ==  ==  ==  ==  ==  ==  ==  ==  ==
