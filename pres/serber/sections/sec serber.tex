% \input{\pathsections "sec serber"}

\section{Liquid Drop Model: Spherical}
% https://tex.stackexchange.com/questions/240243/getting-gif-and-or-moving-images-into-a-latex-presentation
%\begin{frame}\frametitle{Ansatz: Nucleus Behaves as Liquid Drop}
%% By A7N8X - Own work, CC BY-SA 4.0, https://commons.wikimedia.org/w/index.php?curid=48828510
%	\center
%	\animategraphics[loop,controls,width=4cm]{10}{\pLocalGraphics animation/drop-}{0}{81}
%	%\includegraphics{\pLocalGraphics drop.gif}
%\end{frame}

% https://tex.stackexchange.com/questions/240243/getting-gif-and-or-moving-images-into-a-latex-presentation
%\begin{frame}\frametitle{Ansatz: Nucleus Behaves as Liquid Drop}
%        \transduration<0-81>{0}
%        \multiinclude[<+->][format=png, graphics={width=\textwidth}]{\pLocalGraphics animation/drop}
%\end{frame}


%%%   %%%   %%%   %%%
\subsection{Serber's First Principles}
%
\begin{frame}\frametitle{Summary of Liquid Drop Model}
\begin{enumerate}
	\item Nucleus is liquid drop of \bl{constant radius}
	\item Fluid is incompressible
	\item Volume is conserved \cite{hofstadter1956atomic}
\end{enumerate}
\end{frame}
%
\begin{frame}\frametitle{Serber's Ingredients}
\center
	\href{https://www.ucpress.edu/book/9780520344174/the-los-alamos-primer}{
	\begin{overpic}[ scale = 1.15 ]
	{\pLocalGraphics primer-02}
		%\put(-7,-10){Compute the temperature of the bar as a function of position.}
	\end{overpic}}
\center
...based on a set of five lectures given by R. Serber during the first two weeks of April 1943, as an \href{https://upload.wikimedia.org/wikipedia/commons/9/9c/Los_Alamos_Primer.pdf}{\bl{``indoctrination course''}} in connection with the \bl{starting of the \href{https://www.nps.gov/articles/000/manhattan-project-science-at-los-alamos.htm}{Los Alamos Project}}. 
\end{frame}
%
%
\begin{frame}\frametitle{Serber's Ingredients}
\begin{enumerate}
	\item \href{https://mathworld.wolfram.com/Sphere.html}{Volume of Sphere}
	\item \href{https://www.feynmanlectures.caltech.edu/II_08.html}{High School Electrostatics}
\end{enumerate}
\end{frame}
%

%\setbeamercolor{background canvas}{bg=black}
%\begin{frame}\frametitle{Reaction Model: Charge Bisection}
%\center
%	\begin{overpic}[ scale = 0.5 ]
%	{\pLocalGraphics fission-3d-01}
%		%\put(-7,-10){Compute the temperature of the bar as a function of position.}
%	\end{overpic}
%\end{frame}

\setbeamercolor{background canvas}{bg=transparent}
\setbeamercolor{background canvas}{bg=white}
\begin{frame}\frametitle{Reaction Model: Charge Bisection}
\center
	U$_{92}$ $\to$ Nb$_{41}$ $+$ Nb$_{41}$
\end{frame}

\setbeamercolor{background canvas}{bg=black}
\begin{frame}\frametitle{Reaction Model: Charge Bisection}
\center
{\color{black}{Before}} \qquad \qquad {\color{black}{After}}\\
	\begin{overpic}[ scale = 0.425 ]
	{\pLocalGraphics fission-3d-01}
		\put(21,43){$Z=92$}
		\put(65,60){$Z=41$}
		\put(64,26){$Z=41$}
	\end{overpic}
\end{frame}

%\setbeamercolor{background canvas}{bg=black}
%\begin{frame}\frametitle{Quantify Change In Electrostatic Energy}
%\center
%	\begin{overpic}[ scale = 0.5 ]
%	{\pLocalGraphics fission-3d-01}
%		\put(27,43){$E_{0}$}
%		\put(69,61){$E_{1}$}
%		\put(69,26){$E_{1}$}
%	\end{overpic}
% \end{frame}
%\setbeamercolor{background canvas}{bg=white}

\setbeamercolor{background canvas}{bg=black}
\begin{frame}\frametitle{Quantify Change In \href{https://www.macmillanlearning.com/studentresources/college/physics/tiplermodernphysics6e/classial_concept_review/chapter_11_ccr_20_electrostatic_energy_of_a_sphere_of_charge.pdf}{Electrostatic Energy}\jumpLittle}
\center
{\color{white}{Before}} \qquad \qquad {\color{white}{After}}\\
	\begin{overpic}[ scale = 0.425 ]
	{\pLocalGraphics fission-3d-01}
		\put(27,43){$E_{0}$}
		\put(69,61){$E_{1}$}
		\put(69,26){$E_{1}$}
	\end{overpic}
 \end{frame}
\setbeamercolor{background canvas}{bg=white}
%
\subsection{Computation}
\begin{frame}\frametitle{\electrostatic: Before and After}
	%
\center
Compute \href{http://www.atmo.arizona.edu/students/courselinks/spring13/atmo589/ATMO489_online/lecture_10/potential_energy/charged_sphere_pot_energy.html}{\bl{electrostatic energy}} for both configurations.
	%
\end{frame}

\begin{frame}\frametitle{\electrostatic\ For Sphere of Charge $Z$}
\center
\bl{\electrostatic} For Sphere of Charge $Z$: \\[20pt]
\href{https://physics.stackexchange.com/questions/538094/self-energy-of-a-uniformly-charged-non-conducting-sphere-using-energy-density}{
\begin{equation}
	E = \frac{Z^{2}}{R}
%\label{eq:}
\end{equation}}
\end{frame}


\begin{frame}\frametitle{\electrostatic: Before and After}
\begin{table}[htp]
%\caption{default}
	\begin{center}
	\begin{tabular}{ccc}
		Before & \qquad & After \\\hline
		\ \\
		$\bl{E_{0}} = \displaystyle\frac{92^{2}}{\bl{R_{0}}}$ &&
		$\bl{E_{1}} = 2 \times \displaystyle\frac{41^{2}}{\bl{R_{1}}}$ \\
	\end{tabular}
	\end{center}
\label{tab:electrostatic energy}
\end{table}%
\end{frame}

\begin{frame}\frametitle{\electrostatic: Before and After}
\begin{table}[htp]
%\caption{default}
	\begin{center}
	\begin{tabular}{ccc}
		Before & \qquad & After \\\hline
		\ \\
		$\bl{E_{0}} = \displaystyle\frac{92^{2}}{\bl{R_{0}}}$ &&
		$\bl{E_{1}} = 2 \times \displaystyle\frac{41^{2}}{\bl{R_{1}}}$ \\
		\includegraphics[ width = 1in ]{\pLocalGraphics fission-3d-before} &&
		\includegraphics[ width = 1in ]{\pLocalGraphics fission-3d-after}
	\end{tabular}
	\end{center}
\label{tab:electrostatic energy II}
\end{table}%
\end{frame}

\begin{frame}\frametitle{Change in Radius}
	%
\center
How does \bl{$R_{1}$} compare to \bl{$R_{0}$}? \\[10pt]
\pause
Use \rd{Conservation of Nuclear Fluid}.
	%
\end{frame}

\begin{frame}\frametitle{Nuclear Fluid (Volume) is Conserved}
\begin{table}[htp]
%\caption{default}
	\begin{center}
	\begin{tabular}{ccc}
		Before & \qquad & After \\\hline
		\ \\
		$\bl{E_{0}} = \displaystyle\frac{92^{2}}{\bl{R_{0}}}$ &&
		$\bl{E_{1}} = 2 \times \displaystyle\frac{41^{2}}{\bl{R_{1}}}$ \\[15pt]
		$V_{0} = \frac{4}{3}\pi R_{0}^{3}$ && $V_{1} = 2\times\frac{4}{3}\pi R_{1}^{3}$ \\[5pt]
		\includegraphics[ width = 1in ]{\pLocalGraphics fission-3d-before} &&
		\includegraphics[ width = 1in ]{\pLocalGraphics fission-3d-after} \\
	\end{tabular}
	\end{center}
\end{table}%
\end{frame}

\begin{frame}\frametitle{Solve for Radius $R_{1}$}
	%
	$$V_{0} = V_{1} \quad \Rightarrow \quad R_{0}^{3} = 2R_{1}^{3}$$
	%
\end{frame}

\begin{frame}\frametitle{Solve for Radius $R_{1}$}
\begin{table}[htp]
%\caption{default}
\begin{center}
\begin{tabular}{ccc}
	$V_{0} = V_{1}$  & $\Rightarrow$ & $\cancel{\tfrac{4}{3} \pi} \bl{R_{0}^{3}} = \bl{2}\cdot \cancel{\tfrac{4}{3} \pi} \bl{R_{1}^{3}}$ \\[5pt]
			&& $\Downarrow$ \\[5pt]
			&& $R_{0}^{3} = 2R_{1}^{3}$ \\[5pt]
			&& $\Downarrow$ \\[5pt]
			&& $R_{1} = \bl{2^{-\tfrac{1}{3}}}R_{0}$ \\[5pt]
			&& $\Downarrow$ \\[5pt]
			&& $R_{1} \approx \bl{\tfrac{4}{5}}R_{0}$
\end{tabular}
\end{center}
\label{default}
\end{table}%
	%
\end{frame}
%
%
\begin{frame}\frametitle{Change in Electrostatic Energy}
\begin{equation}
		%
	\bl{\Delta E} = E_{0} - 2E_{1} = \mg{\frac{92^{2}}{R_{0}} - 2 \frac{41^{2}}{R_{1}}} = \bl{\frac{3}{8}E_{0}}
		%
\label{eq:delta E}
\end{equation}
\end{frame}

\begin{frame}\frametitle{Change in Electrostatic Energy}
\begin{equation}
		%
	\bl{\Delta E} = \bl{\frac{3}{8}E_{0}} \qquad \raisebox{-1cm}{\includegraphics[ width = 1 in ]{\pLocalGraphics emoji/surprise-03.jpeg}}
		%
\tag{\ref{eq:delta E}}
\end{equation}
\end{frame}

\begin{frame}\frametitle{Change in Electrostatic Energy: Algebra Details}
\begin{equation}
	\begin{split}
		\bl{\Delta E} &= E_{0} - 2E_{1} \\
			&= \mg{\frac{92^{2}}{R_{0}} - 2 \cdot \frac{41^{2}}{R_{1}}} \\
			&= \mg{\frac{(2\cdot 41)^{2}}{R_{0}} - \frac{2 \times 41^{2}}{\tfrac{4}{5}R_{0}}} \\
			&= \mg{\frac{4\cdot 41^{2} - 5/2 \cdot 41^{2} }{R_{0}}} \\
			&= \mg{\frac{3}{2} \frac{41^{2}}{R_{0}}} \\
			&= \bl{\frac{3}{8}E_{0}}
	\end{split}
%\label{eq:}
\end{equation}
\end{frame}
%
\begin{frame}\frametitle{Serber's Ingredients and Result\jumpLittle}
\begin{enumerate}
	\item \href{https://en.wikipedia.org/wiki/Sphere}{\mg{Volume of Sphere}}
	\item \href{https://www.cliffsnotes.com/study-guides/physics/electricity-and-magnetism/electrostatics}{\mg{High School Electrostatics}}
\end{enumerate}
\ \\[10pt]
\center
Result: \bl{Fission} Releases \bl{Astonishing Energy} 
\begin{equation}
		%
	\bl{\Delta E} = \bl{\frac{3}{8}E_{0}} 
		%
\tag{\ref{eq:delta E}}
\end{equation}
\end{frame}
%
%\begin{frame}\frametitle{Fission Releases Astonishing Energy}
%\center
%Over a \bl{third} of the \bl{electrostatic energy} becomes \bl{kinetic energy}!
%\end{frame}
%%
%%
%\begin{frame}\frametitle{Fission Releases Astonishing Energy}
%\center
%Over a \bl{third} of the \bl{electrostatic energy} becomes \bl{kinetic energy}!
%\pause
%If TNT = \$1, U fusion is a stack of dollar bills of $6\tfrac{3}{4}$ miles high.
%\end{frame}


%%%   %%%   %%%   %%%
%\subsection{Rudiments}

%%%   %%%   %%%   %%%
%\subsection{Serber Example}
%
%%%   %%%   %%%   %%%
\subsection{Fission Energy Release}
\begin{frame}\frametitle{Astonishing Energy Release}
\center
	If \href{https://en.wikipedia.org/wiki/TNT}{TNT} gives you \$1 \href{http://chemsite.lsrhs.net/Nuclear/chemNucDifference.html}{per reaction}, how much will $^{235}$U provide? \\[20pt]
	\pause
	A stack of dollar bills 7,160 ft tall.\\[20pt]
	\pause
	\bl{1.4 miles high.}
\end{frame}

\begin{frame}\frametitle{Astonishing Energy Release}
\center
	\includegraphics[ scale = 0.8 ]{\pLocalGraphics singles}
\end{frame}

\begin{frame}\frametitle{Hiroshima\jumpLittle}
\center
	Amount of $^{235}$U fissioned \bl{$< 1.5$ cubic inches}
	\includegraphics[ scale = 0.3 ]{\pLocalGraphics cube}
\end{frame}

%
%%%   %%%   %%%   %%%
\subsection{References}
\begin{frame}\frametitle{Internet References: Liquid Drop Model\jumpLittle}
\begin{itemize}
	\item nuclear-power.com: \href{https://www.nuclear-power.com/nuclear-power/fission/liquid-drop-model/}{\mg{Liquid Drop Model of Nucleus}}
	\item University of Saskatchewan: \href{http://nucleus.usask.ca/ftp/pub/rob/PHYS-452-Topics/Part-04\%20Liquid\%20Drop\%20Model.pdf}{\mg{Liquid Drop Model}}
	\item Colorado School of Mines: \href{http://inside.mines.edu/~kleach/PHGN422/lectures/Lecture5.pdf}{\mg{The Liquid Drop Model of the Nucleus}}
	\item Wikipedia: \href{https://en.wikipedia.org/wiki/Semi-empirical\_mass\_formula}{\mg{Semi-empirical mass formula}}
	\item University of Southampton: \href{http://www.personal.soton.ac.uk/ab1u06/teaching/phys3002/course/04\_liquiddrop.pdf}{\mg{The Liquid Drop Model}}
	\item Vendatu: \href{https://www.vedantu.com/physics/liquid-drop-model}{\mg{Liquid Drop Model in Nuclear Physics}}
	\item hyperphysics: \href{http://hyperphysics.gsu.edu/hbase/Nuclear/liqdrop.html}{\mg{Liquid Drop Model of Nucleus}}
\end{itemize}
\end{frame}

%%%   %%%   %%%   %%%
\subsection{Next Step}
%
\begin{frame}\frametitle{Next Step: Variable $R$\jumpBig}
	%
Serber's argument used a constant radius: $R=\,$\bl{const}\\
%
Bohr and Wheeler let the radius vary $R=\bl{R(\theta)}$
	%
\end{frame}



\endinput  %  ==  ==  ==  ==  ==  ==  ==  ==  ==
