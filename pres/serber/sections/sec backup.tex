% \input{\pSections "sec backup"}

\section{Backup}

%%%   %%%   %%%   %%%
\subsection{Legendre Polynomials: Multipole Expansion}
\begin{frame}\frametitle{Birth of the \legendremathworld}
``In the study of the attraction of spheroids and \bl{planetary motions}''(\cite[ch. 3, p. 169]{sansone1959orthogonal})\\
%
\begin{equation}
	\frac{1}{r} = \paren{1 - 2\rho \cos \gamma + \rho^{2}}^{-\tfrac{1}{2}}
\label{eq:leg-start}
\end{equation}
\pause
Think: \LawOfCosinesMathworld \rd{\ Again}
\end{frame}

\begin{frame}\frametitle{\binomialSeriesMathworld}
For $\abs{\rho} < 1$
\begin{equation}
	\begin{split}
		\frac{1}{r} &= \paren{1 - \rho e^{i\gamma}}^{-\tfrac{1}{2}} \paren{1 - \rho e^{-i\gamma}}^{\rd{-}\tfrac{1}{2}} \\
			&= \sum_{n=0}^{\infty}\paren{-1}^{n} \binom{-\tfrac{1}{2}}{n} \rho^{n} e^{iny}
	\end{split}
\label{eq:expand 1/r}
\end{equation}
\end{frame}

\begin{frame}\frametitle{Using the \ \binomialSeriesMathworld}
For $\abs{\rho} < 1$
\begin{equation}
\begin{array}{cccc}
		\displaystyle\frac{1}{r} &=& \paren{1 - \rho e^{i\gamma}}^{-\tfrac{1}{2}} & \paren{1 - \rho e^{-i\gamma}}^{\rd{-}\tfrac{1}{2}} \\[10pt]
			&=& \displaystyle\sum_{n=0}^{\infty}\paren{-1}^{n} \binom{-\tfrac{1}{2}}{n} \rho^{n} e^{in\gamma}
			& \displaystyle\sum_{n=0}^{\infty}\paren{-1}^{n} \binom{-\tfrac{1}{2}}{n} \rho^{n} e^{\rd{-}in\gamma}
\end{array}
\label{eq:expand 1/r}
\end{equation}
\end{frame}

%
\begin{frame}\frametitle{Using the \cauchyProductMathworld}

\begin{center}
	Multiplication of two absolutely convergent series \\
	when $n>0$
\end{center}
	%
\begin{equation*}
	\paren{f \circ g}(n) = \sum_{k=0}^{n} f(n)g(n-k)
%\label{eq:}
\end{equation*}
\end{frame}

\begin{frame}\frametitle{Using the \cauchyProductWikipedia}
\begin{equation*}
\begin{split}
			%
		\displaystyle\frac{1}{r} &= \displaystyle\sum_{n=0}^{\infty} \rho^{n} \displaystyle\sum_{j=0}^{\brac{n/2}} 
			\binom{-\tfrac{1}{2}}{j} \binom{-\tfrac{1}{2}}{n-j} \times \\
			& \paren{
				\paren{-1}^{j} \paren{-1}^{n-j} e^{ij\gamma}e^{\rd{-}i(n-j)\gamma}
				+ \paren{-1}^{j} \paren{-1}^{n-j} e^{\rd{-}ij\gamma}e^{i(n-j)\gamma}
			}
			%
\end{split}
\end{equation*}
\end{frame}

%
\begin{frame}\frametitle{Introducing the \legendremathworld!\jumpLittle}
	%
\begin{equation}
		%
	\frac{1}{r} = \paren{1 - 2\rho \cos \gamma + \rho^{2}}^{-\tfrac{1}{2}} = \displaystyle\sum_{n=0}^{\infty} \rho^{n} \bl{P_{n}}\paren{\cos \gamma}
		%
\tag{\ref{eq:leg-start}}
\end{equation}
\pause
\begin{equation*}
\begin{split}
		%
	&\bl{P_{n}}\paren{\cos \gamma} \\
		&= \paren{-1}^{n} \paren{\binom{-\tfrac{1}{2}}{n} 2 \cos n \gamma + \binom{-\tfrac{1}{2}}{n-1} \binom{-\tfrac{1}{2}}{1}  2 \cos (n-2) \gamma + \dots}
		%
\end{split}
%\label{eq:}
\end{equation*}
\end{frame}
%



%%%   %%%   %%%   %%%
\subsection{Legendre Polynomials: Bonnet Recursion}
\begin{frame}\frametitle{Deriving Bonnet's Recursion Formula}
Start with the \href{https://en.wikipedia.org/wiki/Legendre\_polynomials\#Definition\_via\_generating\_function}{\bl{generating function}}
\begin{equation}
	\frac{1}{\sqrt{1 - 2xt + t^{2}}} = \displaystyle\sum_{k=0}^{\infty}{\legendreTermA}
\label{eq:legendre-generating functions}
\end{equation}
\bl{Differentiate}  w.r.t.:
\begin{equation}
	\begin{array}{ccc}
		\partial_{t} \displaystyle\paren{\frac{1}{\sqrt{1 - 2xt + t^{2}}}} &=& \partial_{t} \paren{ \displaystyle\sum_{k=0}^{\infty}{\legendreTermA} } \\[15pt]
		\displaystyle\frac{x-t}{\paren{\legendreTerm}^{3/2}}  &=& \displaystyle\sum_{k=1}^{\infty}{\legendreTermB} \\
	\end{array}
%\label{eq:name}
\end{equation}
\end{frame}

\begin{frame}\frametitle{Continuing...}
Multiply both sides by the polynomial term...
\begin{equation}
	\begin{array}{rcl}
		\displaystyle\frac{x-t}{\paren{\legendreTerm}^{\rd{3/2}}}  &=& \displaystyle\sum_{k=1}^{\infty}{\legendreTermB} \\
		\pause
		\displaystyle\frac{\mg{x-t}}{\sqrt{\legendreTerm}} &=& \paren{1 - 2xt + t^{2}}\mg{\displaystyle\sum_{k=1}^{\infty}{\legendreTermB}} \\[15pt]
	\end{array}
%\label{eq:name}
\end{equation}
%
\pause
Substitute using equation \eqref{eq:legendre-generating functions}
\begin{equation}
	\begin{array}{rcl}
		\displaystyle\paren{x-t}\bl{\sum_{k=0}^{\infty}\legendreTermA} &=& \paren{1 - 2xt + t^{2}}\displaystyle\sum_{k=1}^{\infty}{\legendreTermB}
	\end{array}
\label{eq:form-a}
\end{equation}
\end{frame}

%\begin{frame}\frametitle{Collect Powers of $t$}
%\begin{equation}
%	\begin{array}{rcl}
%		%
%	\displaystyle\paren{x-t}\sum_{k=0}^{\infty}\legendreTermA &=& \paren{1 - 2xt + t^{2}}\displaystyle\sum_{k=1}^{\infty}{\legendreTermB}\\[10pt]
%		%
%%	\displaystyle\sum_{k=0}^{\infty}{xP_{k}(x)t^{k}} &=&
%%		\paren{2xt - 1} \displaystyle\sum_{k=1}^{\infty}{\legendreTermB} +\displaystyle\paren{x-t}\sum_{k=0}^{\infty}\legendreTermA \\[10pt]
%%		%
%%	t^{2}\displaystyle\sum_{k=1}^{\infty}{\legendreTermB} &=&
%%		2x \displaystyle\sum_{k=1}^{\infty}{\legendreTermC} +\displaystyle\paren{x-t}\sum_{k=0}^{\infty}\legendreTermA - \displaystyle\sum_{k=1}^{\infty}{\legendreTermB}
%		%
%	\end{array}
%\end{equation}	
%%
%\begin{equation}
%	%\begin{split}
%	\displaystyle\sum_{k=0}^{\infty}{xP_{k}(x)t^{k}} - \sum_{k=0}^{\infty}{P_{k}(x)t^{k+1}} 
%		= \sum_{k=0}^{\infty}{k P_{k}(x)t^{k-1}}
%		- \sum_{k=0}^{\infty}{2x P_{k}(x)t^{k}}
%		+ \sum_{k=0}^{\infty}{k P_{k}(x)t^{k+1}}
%	%\end{split}
%%\label{eq:name}
%\end{equation}
%
%\end{frame}%

\begin{frame}\frametitle{Distribute Powers of $t$}
\begin{equation}
	\begin{array}{rcl}
		\displaystyle\paren{x-t}\sum_{k=0}^{\infty}\legendreTermA &=& \paren{1 - 2xt + t^{2}}\displaystyle\sum_{k=1}^{\infty}{\legendreTermB}
	\end{array}
\tag{\ref{eq:form-a}}
\end{equation}
Distribute powers of $t$
\pause
\begin{equation}
	\begin{array}{ccccc c}
			%
		\displaystyle\sum_{k=0}^{\infty}{xP_{k}(x)t^{\bl{k}}} 
			&-&
		\displaystyle\sum_{k=0}^{\infty}{P_{k}(x)t^{\rd{k+1}}} 
			&=& \\[15pt]
		\displaystyle\sum_{k=0}^{\infty}{k P_{k}(x)t^{k-1}}
			&-&
		\displaystyle\sum_{k=0}^{\infty}{2kx P_{k}(x)t^{\bl{k}}}
			&+&
		\displaystyle\sum_{k=0}^{\infty}{k P_{k}(x)t^{\rd{k+1}}}
			%
	\end{array}
\label{eq:same-damn-thing}
\end{equation}	
\end{frame}

%\begin{frame}\frametitle{Collect Powers of $t$}
%\begin{equation}
%	\begin{array}{ccccc c}
%			%
%		\displaystyle\sum_{k=0}^{\infty}{xP_{k}(x)t^{\bl{k}}} 
%			&-&
%		\displaystyle\sum_{k=0}^{\infty}{P_{k}(x)t^{\rd{k+1}}} 
%			&=& \\[15pt]
%		\displaystyle\sum_{k=0}^{\infty}{k P_{k}(x)t^{k-1}}
%			&-&
%		\displaystyle\sum_{k=0}^{\infty}{2kx P_{k}(x)t^{\bl{k}}}
%			&+&
%		\displaystyle\sum_{k=0}^{\infty}{k P_{k}(x)t^{\rd{k+1}}}
%			%
%	\end{array}
%%\label{eq:name}
%\end{equation}	
%\pause
%\begin{equation}
%	\legendreBonnetK
%\end{equation}	
%\end{frame}

\begin{frame}\frametitle{Collect Powers of $t$}
\begin{equation}
	\begin{array}{ccccc c}
			%
		\displaystyle\sum_{k=0}^{\infty}{xP_{k}(x)t^{\bl{k}}} 
			&-&
		\displaystyle\sum_{k=0}^{\infty}{P_{k}(x)t^{\rd{k+1}}} 
			&=& \\[15pt]
		\displaystyle\sum_{k=0}^{\infty}{k P_{k}(x)t^{k-1}}
			&-&
		\displaystyle\sum_{k=0}^{\infty}{2x P_{k}(x)t^{\bl{k}}}
			&+&
		\displaystyle\sum_{k=0}^{\infty}{k P_{k}(x)t^{\rd{k+1}}}
			%
	\end{array}
\tag{\ref{eq:same-damn-thing}}
\end{equation}	
Rearrangement..
\begin{equation}
	\begin{array}{ccccc c}
			%
		\displaystyle\sum_{k=0}^{\infty}{k P_{k}(x)t^{\rd{k+1}}}
			&+&
		\displaystyle\sum_{k=0}^{\infty}{P_{k}(x)t^{\rd{k+1}}} 
			&=& \\[15pt]
		\displaystyle\sum_{k=0}^{\infty}{2kx P_{k}(x)t^{\bl{k}}}
			&+&
		\displaystyle\sum_{k=0}^{\infty}{kxP_{k}(x)t^{\bl{k}}} 
			&-& 
		\displaystyle\sum_{k=0}^{\infty}{k P_{k}(x)t^{k-1}}
			%
	\end{array}
\tag{\ref{eq:same-damn-thing}}
\end{equation}	
\end{frame}

\begin{frame}\frametitle{Collect Powers of $t$}
\begin{equation}
	\begin{array}{ccccc c}
			%
		\displaystyle\sum_{k=0}^{\infty}{k P_{k}(x)t^{\rd{k+1}}}
			&+&
		\displaystyle\sum_{k=0}^{\infty}{P_{k}(x)t^{\rd{k+1}}} 
			&=& \\[15pt]
		\displaystyle\sum_{k=0}^{\infty}{2kx P_{k}(x)t^{\bl{k}}}
			&+&
		\displaystyle\sum_{k=0}^{\infty}{kxP_{k}(x)t^{\bl{k}}} 
			&-& 
		\displaystyle\sum_{k=0}^{\infty}{k P_{k}(x)t^{k-1}}
			%
	\end{array}
\tag{\ref{eq:same-damn-thing}}
\end{equation}	
Group powers of $t$
\begin{equation}
	\displaystyle\sum_{k=0}^{\infty}(k+1)P_{k}(x)t^{\rd{k+1}} 
		= \displaystyle\sum_{k=0}^{\infty}(2k+1) x P_{k}(x)t^{\bl{k}} 
		- \displaystyle\sum_{k=0}^{\infty}{k P_{k}(x)t^{k-1}}
\label{eq:bonnet-almost}
\end{equation}
\end{frame}

\begin{frame}\frametitle{When Are Series Equal?}
\begin{definition}
Two summable infinite series
	$$ \sum_{k=0}^{\infty} a_{k} < \infty, \sum_{k=0}^{\infty} b_{k} < \infty$$
	are equal iff they exhibit equality of the partial sums:
\begin{equation*}
	\sum_{k=0}^{n} a_{k} = \sum_{k=0}^{n} b_{k} \quad \forall n \ge 0
%\label{eq:name}
\end{equation*}
\end{definition}
The series are equal \bl{term by term}:
$$ a_{k} =  b_{k} \quad k=0\colon \infty$$
\end{frame}

\begin{frame}\frametitle{Equate Coefficients Term by Term}
``Eliminate'' the summations
\begin{equation}
	\displaystyle\sum_{k=0}^{\infty}(k+1)P_{k}(x)t^{\rd{k+1}} 
		= \displaystyle\sum_{k=0}^{\infty}(2k+1) x P_{k}(x)t^{\bl{k}} 
		- \displaystyle\sum_{k=0}^{\infty}{k P_{k}(x)t^{k-1}}
\tag{\ref{eq:bonnet-almost}}
\end{equation}
Revealing \bl{Bonnet's Recursion Formula}
\begin{equation}
	\legendreBonnetK
\end{equation}		
\end{frame}


\endinput  %  ==  ==  ==  ==  ==  ==  ==  ==  ==
