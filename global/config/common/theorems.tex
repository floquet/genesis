% \input{\pGlobalConfigCommon/"theorems.tex"}
\typeout{  ==  ==  ==  ==  ==  \pGlobalConfigCommon/"theorems.tex"}

% File: theorems.tex
% Purpose: Unified theorem configuration for documents and presentations.
% Author: Achates, with guidance from Daniel Topa
% Sun Dec 22 21:33:42 MST 2024

% The amsthm package provides tools for defining and formatting theorem-like 
% environments, such as theorems, lemmas, corollaries, and proofs. It ensures 
% consistent numbering and styling of logical statements in mathematical documents.
\usepackage{amsthm}

% Define global theorem styles
% In Section 2, the numbering restarts, e.g., 2.1, 2.2.
% Environments like lemma and corollary share numbering with theorem because of [theorem].
\theoremstyle{plain} % Default style
\newtheorem{theorem}{Theorem}[section]
\newtheorem{lemma}[theorem]{Lemma}
\newtheorem{corollary}[theorem]{Corollary}

% Definition style
\theoremstyle{definition}
\newtheorem{definition}{Definition}[section]
\newtheorem{example}[definition]{Example}

% Remark style
\theoremstyle{remark}
\newtheorem{remark}{Remark}[section]

% Beamer-specific customization
\ifdefined\beamer@rendering
    % Customizations for splitting theorems in Beamer
    \newcommand*{\theorembreak}{%
        \usebeamertemplate{theorem end}%
        \framebreak%
        \usebeamertemplate{theorem begin}%
    }
\else
    % No-op for non-Beamer documents
    \newcommand*{\theorembreak}{}
\fi

% Usage:
% In Beamer, use \theorembreak to split a theorem across frames:
% \begin{theorem}
% This theorem spans multiple frames.
% \theorembreak
% Here is the continuation of the theorem.
% \end{theorem}

\endinput  %  -  -  -  -  -  -  -  -  -  -  -  -  -  -  -  -  -  -  -  -
