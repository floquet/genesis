% \input{\pGlobalConfigCommon/"packages-common.tex"}
\typeout{  ==  ==  ==  ==  ==  \pGlobalConfigCommon/"packages-common.tex"}

% Global setup for packages used in most reports
% The amssymb package provides additional mathematical symbols
% such as \mathbb for blackboard bold (e.g., \mathbb{R} for ℝ),
% and specialized symbols like \nleq, \rightsquigarrow, and \boxdot.
% Often used alongside amsmath for a complete mathematical toolkit.

% ========== Mathematics and Formatting ==========
\usepackage{amsmath}    % Advanced math typesetting
\usepackage{amssymb}    % Extended math symbols
%\usepackage{amsthm}     % Theorem-like environments
\usepackage{mathtools}  % Extensions to amsmath
\usepackage[T1]{fontenc} % Use T1 font encoding for better hyphenation, accented characters, and PDF searchability
\usepackage{lmodern}    % Optional: Load modern Latin fonts for improved appearance and scalability

% ========== Tables ==========
\usepackage{colortbl}   % Add color to table rows and columns
\usepackage{multirow}   % Merge rows in tables
\usepackage{multicol}   % Columns in documents
\usepackage{booktabs}   % Professional table formatting

% ========== Graphics and Overlays ==========
\usepackage{graphicx}   % Core package for image handling
\usepackage{overpic}    % Overlay graphics with LaTeX annotations
\usepackage{tikz}       % Create diagrams and illustrations
    \usetikzlibrary{arrows.meta} % Enhanced arrow styles
    \usetikzlibrary{shapes.geometric} % Geometric shapes
    \usetikzlibrary{positioning} % Relative positioning of nodes
    \usetikzlibrary{decorations.pathmorphing} % Path decorations (e.g., snake lines)

% ========== Miscellaneous ==========
\usepackage{xcolor}     % Extended color definitions
\usepackage{pdfpages}   % Include pages from external PDFs (optional, documents only)

% ========== Hyperlinks ==========
\usepackage{hyperref}   % Hyperlinks and cross-references
\PassOptionsToPackage{hidelinks}{hyperref} % Option to hide link borders
% Note: Load hyperref last in the document to avoid conflicts.

\endinput  %  -  -  -  -  -  -  -  -  -  -  -  -  -  -  -  -  -  -  -  -

Legacy:
    \usetikzlibrary{arrows}      % Basic arrow styles
